\documentclass[10pt, a4paper,spanish]{article}

\usepackage[utf8]{inputenc}
\usepackage[spanish]{babel}

\usepackage{hyperref}

\usepackage[T1]{fontenc}

\usepackage[hmarginratio=1:1,top=32mm,columnsep=20pt]{geometry}
\usepackage[hang, small,labelfont=bf,up,textfont=it,up]{caption}


\usepackage{graphicx}
\graphicspath{ {images/} }
\usepackage{capt-of}% or \usepackage{caption}

\usepackage{abstract}
\renewcommand{\abstractnamefont}{\normalfont\bfseries}
\renewcommand{\abstracttextfont}{\normalfont\small\itshape}

\usepackage{titlesec}
\renewcommand\thesection{\Roman{section}}
\renewcommand\thesubsection{\Roman{subsection}}
\titleformat{\section}[block]{\large\scshape\centering}{\thesection.}{1em}{}
\titleformat{\subsection}[block]{\large}{\thesubsection.}{1em}{}


\usepackage{fancyhdr}
\pagestyle{fancy}
\fancyhead{}
\fancyfoot{}
\fancyhead[C]{ \today \ $\bullet$ Minería de Datos $\bullet$ Comparación entre J48(C4.5) y Naive Bayes}
\fancyfoot[RO]{\thepage}

%-------------------------------------------------------------------------------
%	TITLE SECTION
%-------------------------------------------------------------------------------

\title{\vspace{-15mm}\fontsize{24pt}{10pt}\selectfont\textbf{Comparación entre J48(C4.5) y Naive Bayes}} % Article title

\author{Sergio García Prado}
\date{\today}

%-------------------------------------------------------------------------------

\begin{document}

	\maketitle % Insert title

	\thispagestyle{fancy} % All pages have headers and footers

%-------------------------------------------------------------------------------
%	TEXT
%-------------------------------------------------------------------------------

	\section{Introducción}
        \paragraph{}
		La práctica consiste en la evaluación de dos algoritmos de clasificación aplicando los tests de McNemar y Student. Los algoritmos a examinar son los siguientes:

		\begin{itemize}

			\item \textbf{J48}: es un algoritmo cuya función es generar un árbol de decisión. Su nombre original es \emph{C4.5} pero la implementación de Weka se denomina \emph{J48}. Para generar el árbol de decisión el árbol basa la selección de atributos en cada nodo según la entropía de los mismos con respecto a la clases en la que se desea clasificar las muestras. Es por tanto un algoritmo de aprendizaje supervisado.

			\item \textbf{Naive Bayes}: es un algoritmo de clasificación probabilístico basado en el teorema de Bayes y algunas hipótesis adicionales que facilitan la simplificación del problema. Estas hipótesis presuponen la independencia entre variables, de ahí es de donde proviene el apelativo \emph{naive (ingenuo)} ya que esta presuposición no siempre es cierta. Es un clasificador que utiliza aprendizaje supervisado y utiliza el método de máxima verosimilitud

		\end{itemize}

	\section{Test de McNemar: HoldOut de $2/3$}

        \paragraph{}
		$h_A$ es J48 y $h_B$ es Naive Bayes

		\paragraph{}
		\begin{center}
			\begin{tabular}{ | p{6cm} | p{6cm} | }
				\hline
					Número de ejemplos mal clasificados por $h_A$  y $h_B$ ($n_{00}$) &
					Número de ejemplos mal clasificados por $h_A$  pero no por $h_B$ ($n_{01}$) \\ \hline

					Número de ejemplos mal clasificados por $h_B$ pero no por $h_A$ ($n_{10}$) &
					Número de ejemplos bien clasificados por $h_A$  y $h_B$ ($n_{11}$)\\
				\hline
			\end{tabular}
		\end{center}


		\[
		\frac{(|n_{01}-n{10}|-1)^2}{n_{01}+n_{10}}
		\]


		\begin{table}[ht]
			\begin{minipage}[b]{0.3\linewidth}
				\centering
				\begin{tabular}{ | c | c | }
					\hline
					9 & 9 \\ \hline
					11 & 204 \\
					\hline
				\end{tabular}
				\caption{Soybean}
			\end{minipage}\hfill
			\begin{minipage}[b]{0.3\linewidth}
				\centering
				\begin{tabular}{ | c | c | }
					\hline
					4 & 4 \\ \hline
					13 & 127 \\
					\hline
				\end{tabular}
				\caption{Vote}
			\end{minipage}\hfill
			\begin{minipage}[b]{0.3\linewidth}
				\centering
				\begin{tabular}{ | c | c | }
					\hline
					2 & 2 \\ \hline
					0 & 16 \\
					\hline
				\end{tabular}
				\caption{Labor}
			\end{minipage}
		\end{table}

	\section{Test de Student: Cross Validation sin repetición}

        \paragraph{}

	\section{Test de Student: Cross Validation con repetición}

	        \paragraph{}
	\section{Resultados}

		\paragraph{}
		Los resultados obtenidos según los test realizados con los conjuntos de datos y los tipos de test son los siguientes:

		\paragraph{}
		\begin{center}
			\begin{tabular}{ | c || c | c | c | c | c | }
				\hline
				 			& McNemar	& Student	& Student (C) 	& Student rep. 	& Student rep.(C) 	\\ \hline \hline
				Soybean 	& J48 		& NB		& NB			& NB			& NB			 	\\ \hline
				Vote 		& J48 		& J48		& J48			& J48			& J48				\\ \hline
				Labor 		& NB 		& NB		& NB			& NB			& NB				\\
				\hline
			\end{tabular}
		\end{center}



\end{document}
