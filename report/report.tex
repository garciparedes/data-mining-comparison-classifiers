\documentclass[10pt, a4paper,spanish]{article}

\usepackage[utf8]{inputenc}
\usepackage[spanish]{babel}

\usepackage{hyperref}

\usepackage[T1]{fontenc}

\usepackage[hmarginratio=1:1,top=32mm,columnsep=20pt]{geometry}
\usepackage[hang, small,labelfont=bf,up,textfont=it,up]{caption}


\usepackage{graphicx}
\graphicspath{ {images/} }

\usepackage{abstract}
\renewcommand{\abstractnamefont}{\normalfont\bfseries}
\renewcommand{\abstracttextfont}{\normalfont\small\itshape}

\usepackage{titlesec}
\renewcommand\thesection{\Roman{section}}
\renewcommand\thesubsection{\Roman{subsection}}
\titleformat{\section}[block]{\large\scshape\centering}{\thesection.}{1em}{}
\titleformat{\subsection}[block]{\large}{\thesubsection.}{1em}{}


\usepackage{fancyhdr}
\pagestyle{fancy}
\fancyhead{}
\fancyfoot{}
\fancyhead[C]{ \today \ $\bullet$ Minería de Datos $\bullet$ Comparación entre J48(C4.5) y Naive Bayes}
\fancyfoot[RO]{\thepage}

%-------------------------------------------------------------------------------
%	TITLE SECTION
%-------------------------------------------------------------------------------

\title{\vspace{-15mm}\fontsize{24pt}{10pt}\selectfont\textbf{Comparación entre J48(C4.5) y Naive Bayes}} % Article title

\author{Sergio García Prado}
\date{\today}

%-------------------------------------------------------------------------------

\begin{document}

	\maketitle % Insert title

	\thispagestyle{fancy} % All pages have headers and footers

%-------------------------------------------------------------------------------
%	ABSTRACT
%-------------------------------------------------------------------------------

	\begin{abstract}
		\noindent
	\end{abstract}

%-------------------------------------------------------------------------------
%	TEXT
%-------------------------------------------------------------------------------

	\section{Introducción}
        \paragraph{}


	\section{McNemar}

        \paragraph{}
		$h_A$ es J48 y  $h_B$ es Naive Bayes

		\center
		\begin{tabular}{ | p{6cm} | p{6cm} | }
			\hline
				Número de ejemplos mal clasificados por $h_A$  y $h_B$ ($n_{00}$) &
				Número de ejemplos mal clasificados por $h_A$  pero no por $h_B$ ($n_{01}$) \\ \hline

				Número de ejemplos mal clasificados por $h_B$ pero no por $h_A$ ($n_{10}$) &
				Número de ejemplos bien clasificados por $h_A$  y $h_B$ ($n_{11}$)\\
			\hline
		\end{tabular}


		\subsection{SoyBean}

			\paragraph{}

			\center
			\begin{tabular}{ | l | l | }
				\hline
				9 & 9 \\ \hline
				11 & 204 \\
				\hline
			\end{tabular}


		\subsection{Vote}

			\paragraph{}

			\begin{tabular}{ | l | l | }
				\hline
				4 & 4 \\ \hline
				13 & 127 \\
				\hline
			\end{tabular}

		\subsection{Labor}

			\paragraph{}

			\begin{tabular}{ | l | l | }
				\hline
				2 & 2 \\ \hline
				0 & 16 \\
				\hline
			\end{tabular}

	\section{T-Student}

        \paragraph{}



\end{document}
